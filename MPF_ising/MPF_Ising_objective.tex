\documentclass{article} 
\newcommand{\set}[1]{\lbrace #1 \rbrace}
\newcommand{\setc}[2]{\lbrace #1 \mid #2 \rbrace}
\newcommand{\vv}[1]{{\mathbf{#1}}}
\newcommand{\dd}{{\mathrm{d}}}
\newcommand{\pd}[2]{\frac{\partial #1}{\partial #2}}
\newcommand{\pdn}[3]{\frac{\partial^#1 #2}{\partial #3^#1}}
\newcommand{\od}[2]{\frac{\dd #1}{\dd #2}}
\newcommand{\odn}[3]{\frac{\dd^#1 #2}{\dd #3^#1}}
\newcommand{\avg}[1]{\left< #1 \right>}
\newcommand{\mb}{\mathbf}
\newcommand{\argmin}{\operatornamewithlimits{argmin}}

\usepackage{times}
\usepackage{amsmath,amssymb}
\usepackage{graphicx}

\title{MPF objective function for an Ising model}

\author{
Jascha Sohl-Dickstein
}

\newcommand{\fix}{\marginpar{FIX}}
\newcommand{\new}{\marginpar{NEW}} 

\begin{document}

\maketitle

This document derives the MPF objective function for the case of an Ising model, and connections to all states which differ by a single bit flip.  It is written quickly, and not well proofread!!  Typos and missing steps are likely...

For the Ising model, the energy function is
\begin{align}
E = \mb x^T \mb J \mb x
\end{align}
where $\mb x \in \left\{ 0, 1 \right\}^N$, $\mb J \in \mathcal R^{N\times N}$, and $\mb J$ is symmetric ($\mb J = \mb J^T$).
The MPF objective function is
\begin{align}
K = \sum_{\mb x \in \mathcal D} \sum_n \exp\left( \frac{1}{2}\left[
E(\mb x) - E(\mb x + {\mb d}(\mb x, n)) \right] \right)
\end{align}
where the sum over $d$ indicates a sum over all data dimensions, and the function ${\mb d}(\mb x, n) \in \left\{ -1, 0, 1 \right\}^N$ is
\begin{align}
{\mb d}(\mb x, n)_i =
	\left\{\begin{array}{ccc}
0 & & i \neq n \\
-(2 x_i - 1) & & i = n
	\end{array}\right.
\end{align}

For the Ising model, the MPF objective function becomes
\begin{align}
K & = \sum_{\mb x \in \mathcal D} \sum_n \exp\left( \frac{1}{2}\left[
\mb x^T \mb J \mb x
- (\mb x + {\mb d}(\mb x, n))^T \mb J (\mb x + {\mb d}(\mb x, n))
\right] \right) \\
& = \sum_{\mb x \in \mathcal D} \sum_n \exp\left( \frac{1}{2}\left[
\mb x^T \mb J \mb x
- \left(
\mb x^T \mb J \mb x
+
2 \mb x^T \mb J {\mb d}(\mb x, n)
+
{\mb d}(\mb x, n)^T \mb J {\mb d}(\mb x, n)
\right) 
\right] \right)
 \\
& = \sum_{\mb x \in \mathcal D} \sum_n \exp\left( -\frac{1}{2}\left[
2 \mb x^T \mb J {\mb d}(\mb x, n)
+
{\mb d}(\mb x, n)^T \mb J {\mb d}(\mb x, n)
\right]
\right)  \\
& = \sum_{\mb x \in \mathcal D} \sum_n \exp\left( -\frac{1}{2}\left[
2 \sum_i x_i J_{in} \left( 1 - 2 x_n  \right)
+
J_{nn}
\right]
\right)\\
& = \sum_{\mb x \in \mathcal D} \sum_n \exp\left( \left[
\left( 2 x_n - 1 \right) \sum_i x_i J_{in}
-
\frac{1}{2}J_{nn}
\right]
\right)
.
\end{align}
Ignoring the symmetry constraint on $\mb J$, the derivative is
\begin{align}
\left[ \pd{K}{J_{lm}} \right]^{\mathrm{asymmetric}} & = \sum_{\mb x \in \mathcal D} \exp\left( \left[
\left( 2 x_m - 1 \right) \sum_i x_i J_{im}
-
\frac{1}{2}J_{mm}
\right]
\right)
%\times \nonumber \\ & \qquad \qquad
	\left[
		\left( 2 x_m - 1 \right) x_l
		-
		\delta_{lm} \frac{1}{2}
	\right]
.
\end{align}
Enforcing symmetry, the derivative is
\begin{align}
\pd{K}{J_{lm}}  = \frac{1}{2}\left[ \pd{K}{J_{lm}} \right]^{\mathrm{asymmetric}} +\frac{1}{2}\left[ \pd{K}{J_{ml}} \right]^{\mathrm{asymmetric}}
.
\end{align}

Note that both the objective function and gradient can be calculated using matrix operations (no for loops).  See the code.
		
\end{document}
