\documentclass{article} 
\newcommand{\set}[1]{\lbrace #1 \rbrace}
\newcommand{\setc}[2]{\lbrace #1 \mid #2 \rbrace}
\newcommand{\vv}[1]{{\mathbf{#1}}}
\newcommand{\dd}{{\mathrm{d}}}
\newcommand{\pd}[2]{\frac{\partial #1}{\partial #2}}
\newcommand{\pdn}[3]{\frac{\partial^#1 #2}{\partial #3^#1}}
\newcommand{\od}[2]{\frac{\dd #1}{\dd #2}}
\newcommand{\odn}[3]{\frac{\dd^#1 #2}{\dd #3^#1}}
\newcommand{\avg}[1]{\left< #1 \right>}

\newcommand{\mb}{\mathbf}
\newcommand{\argmin}{\operatornamewithlimits{argmin}}
\newcommand{\ham}[2]{\operatornamewithlimits{HAM}\left( #1 ; #2 \right)}


\usepackage{times}
\usepackage{amsmath,amssymb}
\usepackage{graphicx}

\title{MPF objective function for an RBM}

\author{
Jascha Sohl-Dickstein
}

\newcommand{\fix}{\marginpar{FIX}}
\newcommand{\new}{\marginpar{NEW}} 

\begin{document}

\maketitle

This document derives the MPF objective function for the case of an RBM, and connections to all states which differ by a single bit flip. % It is written quickly, and not well proofread!!  Typos and missing steps are likely...

The energy function over the visible
units for an RBM is found by marginalizing out the hidden units.  This
gives an energy function of:
\begin{align}
E(\mb x) &= -\sum_i \log ( 1 + \exp ( -W_i \mb x ) )
\end{align}
where $W_i$ is a vector of coupling parameters and $\mb x$ is the binary input
vector.  The MPF objective function for this is
\begin{align}
K &= \sum_{\mb x \in \mathcal D} \sum_n \exp\left( \frac{1}{2}\left[
E(\mb x) - E(\mb x + {\mb d}(\mb x, n)) \right] \right)
\end{align}
where the sum over $n$ indicates a sum over all data dimensions, and
the function ${\mb d}(\mb x, n)$ is
\begin{align}
{\mb d}(\mb x, n)_i =
	\left\{\begin{array}{ccc}
0 & & i \neq n \\
-(2 x_i - 1) & & i = n
	\end{array}\right.
\end{align}

Substituting into the objective function
\begin{align}
K = \sum_{\mb x \in \mathcal D} \sum_n \exp\left( \frac{1}{2}\left[
-\sum_i \log ( 1 + \exp ( -W_i \mb x ) )
+
\sum_i \log ( 1 + \exp ( -W_i \left(  \mb x + {\mb d}(\mb x, n) \right) ) )
\right] \right)
\end{align}

The released matlab code implements the sum over $n$ in a for loop, and
calculates the change in $W_i \mb x$ caused by $W_i {\mb d}( \mb x, n)$ for all samples simultaneously.  (This is a minor savings - only one column of $W$ needs to be visited).  (Note that the for loop could also
be done over samples, with the change induced by each bit flip being
calculated by matrix operations.  If the code is run with a small
batch size, this implementation would be faster.)  Though I could not
see a way to replace both for loops with matrix operations, a more
clever programmer might!
		
\end{document}
